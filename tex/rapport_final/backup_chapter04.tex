% Chapitre 4 : travaux sur la grille de voxels de noura
\chapter{Analyse de volume et génération de grilles}\label{chapter04:vispaper}
{
	\commentaire{
		\begin{enumerate}
			% Recherche de voisins {{{
			\item travaux sur la recherche de voisins\begin{enumerate}
				\item introduction au problème : à tulane ils le font pas, et on va le faire pour eux
				\item rappel sur les paramètres du microscope
				\item présentation graphique du problème des voisins
				\item explication sur le besoin d'une méthode générique / paramétrisable
				\item explication de la méthode : espaces différents, relations mathématiques entre eux ... (cf le dessin de benjamin)
				\item implémentation de cette méthode (sans code, rapide, montrer aspect modulaire à 'n' stacks)
				\item présentation visuelle \todo{faire screenshots de l'appli en dual-visu}
				\item transition : on peut mieux avoir les voisins et donc interpoler comme il faut les valeurs, donc grille de voxels !
			\end{enumerate}
			% }}}
			% Génération de grilles {{{
			\item travaux sur la génération de grilles\begin{enumerate}
				\item recherche des voisins faite, mais toujours pas de notion de grille
				\item génération de grille : la facon simple\begin{itemize}
					\item Dump la grille telle quelle après transfo dans l'espace réel
					\item du coup pas de notion de voisins, et beaucoup d'espace réellement vide (pas de données par rapport à donnée nulle)
				\end{itemize}
				\item génération de grille : prendre en compte les voisins, et interpoler pour avoir des résultats cohérents~\begin{itemize}
					\item l'interpolation nearest neighbor : tout bête
					\item l'interpolation trilinéaire : c'est un petit peu mieux, permet d'estimer un blending des valeurs aux points pour mieux estimer la donnée
					\item l'interpolation barycentrique : meme effet que 3-lin., et en + : résiste au changement d'espace car espace défini par position des points du tétraèdre
				\end{itemize}
				\item Note : avec interpolations, présenter avantages/inconvénients de chacuns afin de les comparer
				\item présenter possibilités de faire rendu très détaillé de une sous partie de la chose, afin de mieux voir de quoi il s'agit
				\item le blending des deux stacks : mais pourquoi, et comment ?
				\item implémentation (rapide, vu qu'on en parle extensivement avant)
			\end{enumerate}
			% }}}
		\end{enumerate}
	}

	\section{Recherche de voisins}
	{
		\commentaire{}\\
	}

	\section{Génération de grille}
	{
		\commentaire{}\\
	}
}

% VIM modeline : do not touch !
% vim: set spell spelllang=fr :
