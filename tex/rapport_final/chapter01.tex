% Chapitre 1 : Introduction
\chapter{Introduction}\label{chapter:01:introduction}

\commentaire{
	\begin{enumerate}
		\item introduction au sujet de stage~\begin{itemize}
			\item présentation de la reconstruction 3D
			\item intérêt dans le médical
			\item transition : présentation de la prostate (à revoir)
		\end{itemize}
		\item introduction à la problématique (coupes histographiques, gleason, ...)~\begin{itemize}
			\item prostate : beaucoup de cancers
			\item cancer de la prostate : dur à identifier
			\item pourquoi dur à identifier ? $\rightarrow$ échelle de Gleason
			\item présentation rapide de l'échelle de Gleason
			\item intérêt de la reconstruction 3D : vraie forme des glandes.
			\item (?) intérêt du stage : Tulane ne le fait pas
			\item transition (?)
		\end{itemize}
		\item rapide overview du travail fait~\begin{itemize}
			\item présentation du chargement des images
			\item présentation de la recherche de voisins
			\item présentation de la génération de grille de voxels
		\end{itemize}
	\end{enumerate}
}\par

\wip{
	Ce stage s'inscrit dans un effort de collaboration internationale entre l'université de Tulane à la Nouvelle-Orléans, et l'université de Montpellier.
}

% VIM modeline : do not touch !
% vim: set spell spelllang=fr :
