% Chapitre 1 : Introduction
\chapter{Introduction}\label{chapter:01:introduction}
{
	\commentaire{
		\begin{enumerate}
			\item introduction au sujet de stage~\begin{itemize}
				\item présentation de la reconstruction 3D
				\item intérêt dans le médical
				\item transition : présentation de la prostate (à revoir)
			\end{itemize}
			\item introduction à la problématique (coupes histologiques, Gleason, ...)~\begin{itemize}
				\item prostate : beaucoup de cancers
				\item cancer de la prostate : identifier $\rightarrow$ échelle de Gleason
				\item présentation rapide de l'échelle de Gleason
				\item intérêt de la reconstruction 3D : vraie forme des glandes.
				\item (?) intérêt du stage : Tulane ne le fait pas
				\item transition (?)
			\end{itemize}
			\item rapide overview du travail fait~\begin{itemize}
				\item présentation du chargement des images
				\item présentation de la recherche de voisins
				\item présentation de la génération de grille de voxels
			\end{itemize}
		\end{enumerate}
	}\par

	\wip{La reconstruction 3D permet d'obtenir une représentation tridimensionnelle d'un objet ou d'une scène à partir d'un ensemble de primitives, tells que des points, des images ou des sous-ensembles tridimensionnels. Ces représentations tridimensionnelles des objets permettent de mieux percevoir le fonctionnement et l'utilisation implicite d'un objet grâce aux affordances qu'il nous permet d'inférer.\todo{Vraiment nécessaire ? On sait que c'est mieux en 3D, mais trouver meilleure excuse.}}\\

	\wip{Naturellement, une représentation tridimensionnelle et interactive [...]}\\

	\wip{Ce sujet de stage s'inscrit dans un effort de collaboration internationale entre l'université de Tulane à la Nouvelle Orléans, et l'université de Montpellier dans le cadre d'un projet pour l'aide à la détection et au diagnostic du cancer de la prostate. Ainsi, il est nécessaire de préciser certaines choses sur les caractéristiques de cette pathologie.\\Tout d'abord, selon l'IARC, l'Agence Internationale pour la Recherche sur le Cancer de l'Organisation Mondiale de la Santé, le cancer de la prostate était l'un des types de cancers les plus courants tous sexes confondus en 2018\footnote{Voir : \protect{\url{https://gco.iarc.fr/today/home}}}. Cette même année, plus d'un million nouveaux cas de cancers de la prostate furent diagnostiqués dans le monde.}\\

	\wip{Le meilleur pronostic disponible est une échelle histologique, appelée échelle de Gleason. Celle ci attribue un score, appelé score de Gleason, en se basant sur l'observation des formes des glandes dans une coupe du tissu de la prostate d'un patient. Or, étant une échelle se basant sur l'observation de la forme des glandes dans une coupe histologique, donc effectivement bidimensionnelle, d'un objet tridimensionnel, le biais induit par la direction et l'angle de la coupe influera beaucoup le pronostic final donné par un praticien. Ainsi, cette méthode de pronostic pourrait engendrer de faux pronostics, pouvant soit causer des frais médicaux non nécessaires pour un patient sain, mais aussi pouvant laisser le cancer d'un patient se développer librement, sans aucune action prise à son encontre.}\\

	\wip{Dans ce contexte, une reconstruction 3D permettrait d'effectuer un meilleur pronostic en étudiant la forme réelle des glandes. Le but du projet, à long terme, serait de créer une nouvelle procédure de diagnostic du cancer de la prostate, utilisant l'information contenue dans la forme réelle (tridimensionnelle) des glandes.}\\

}
% VIM modeline : do not touch !
% vim: set spell spelllang=fr :
