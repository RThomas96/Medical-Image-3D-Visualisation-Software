% Chapter 5 : Conclusion, réflection & difficultés rencontrées, ouverture
\chapter{Conclusion}\label{chapter:06:conclusion}
{
    \iffalse
    % plan {{{
	\commentaire{
		\begin{enumerate}
			% conclusion {{{
			\item Conclusion : travaux effectués. Buts originaux :~\begin{itemize}
				\item Gestion de données massives [x] :\begin{itemize}
					\item on peut en effet charger des images massives en mémoire, même si celles ci sont moins massives que les données réelles de Tulane (travail à suivre après le stage)
					\item on peut aussi charger des versions basse résolution des images, au cas où les full-res ne rentrent pas en mémoire (option de downsampling dans le loader)
				\end{itemize}
				\item Exploration interactive [x] :~\begin{itemize}
					\item On peut utiliser des plans de coupe dans l'application en single-view, pour traverser le volume (plans de coupe surement ajoutés dans vue double sous peu, mais peut être après rapport/soutenance)
					\item Techniquement, on pourrait aussi exporter une grille et le voir avec Texture3D
				\end{itemize}
				\item Visualisation multi-résolution~\begin{itemize}
					\item Ce but est plus flou, parce que oui, on peut générer des grilles à plusieurs résolutions et les voir, mais bon, ca casse le côté interactif de la visu ...
				\end{itemize}
			\end{itemize}
			% }}}
			% reflection {{{
			\item Réflection : difficultés encontrées\begin{itemize}
				\item confinement $\rightarrow$ pas de réunion en présentiel et tous les jours se ressemblent et s'assemblent $\rightarrow$ moral au plus bas
				\item exemple : opengl : je me suis pas posé et ai plutot foncé tête baissée $\rightarrow$ perte de temps, d'énergie et de toute envie de quoi que ce soit
				\item problème perso : ne veut pas faire de petits pas, mais plutot un grand en + de tps
				\item ouverture : + de travail sur moi même \& ma facon de travailler (malheureusement, seule ouverture possible dans ce contexte)
			\end{itemize}
			% }}}
			% futur, pour juillet & thèse {{{
			\item Ouverture : possibilités de continuation dans un futur proche\begin{itemize}
				\item En juillet\begin{enumerate}
					\item Implémenter version offline, sans accès graphique
					\item Implémenter version incrémentale et/ou partielle (d'abord recalage, sauvegardé dans un fichier, puis stitching des images en grille)
					\item Faire plutot visu multi res
					\item Et d'autres encore
				\end{enumerate}
			\item Ouverture : thèse ! Travaux à venir :\todo{A garder ?}\begin{enumerate}
					\item Remeshing des glandes pour démarrer le travail de reconstruction approfondie et détaillé (?)
					\item ???
				\end{enumerate}
			\end{itemize}
			% }}}
		\end{enumerate}
	}
	% }}}
	\fi
	
	\section{Contributions}
	{
        Au début de ce stage, nous avions 3 objectifs :~\begin{itemize}
			\item proposer une méthode de gestion de données massives,
			\item proposer une méthode d'exploration interactive de ces données,
			\item et proposer une méthode de visualisation multi-résolution.
        \end{itemize}

        Nous avons pu proposer une première méthode de gestion des données, sur la base des jeux de données que nous ont fournis l'université de Tulane. Cette gestion permet de charger une version sous-échantillonnée d'une image à la volée, afin de limiter l'espace mémoire occupé par les images pour des tâches où avoir le plus d'informations n'est pas nécessaire, comme par exemple la visualisation, où grâce à la résolution spatiale du microscope nous pouvons toujours visualiser la morphologie des glandes sans perte d'informations.

        Nous avons ensuite proposé deux méthodes d'exploration interactive de ces données. Une première méthode par plans de coupe, permettant une traversée interactive des données chargées en mémoire, et une méthode de visualisation par sous domaine facilement extensible à des images non segmentées. Ces deux méthodes d'exploration permettent de manipuler en temps réel une grille de voxels pour des buts d'examens approfondis des captures de prostate.

        Nous avons également proposé une méthode de reconstruction de grille de voxels effectuée à la volée, qui pourra permettre la mise en place d'une architecture de visualisation multi-résolution des acquisitions de prostate. En plus de cela, cette méthode de génération de grilles est générique, afin de permettre la génération de grilles de voxels de taille et résolutions arbitraires. Cette méthode est plus rapide que la méthode que Tulane emploie, et permettrait de recréer des échantillons de prostate en trois dimensions peu de temps après le prélèvement chez un patient.
	}

	\section{Réflexion}
	{
        L'effet qu'a eu le confinement déclaré par l'\'Etat Français afin d'endiguer la pandémie de \textit{COVID-19} ne peut être négligé. Cette période d'isolement social et mental fut compliquée à vivre. Elle a permis de faire ressortir des failles dans mes méthodes de travail, qui sont principalement un manque de capacité à découper un problème en sous-problèmes, ainsi qu'un manque de communication envers mes tuteurs et collègues lorsque nous sommes en distanciel. Le fait de ne pas pouvoir diviser des tâches efficacement est l'une des raisons principales qui m'a poussé à prendre du retard pendant ce stage. Toutefois, une fois ces problèmes identifiés et la période de confinement levée, j'ai pu discuter de ces problèmes avec mes encadrants et nous avons pu travailler sur des méthodes de travail afin d'éliminer progressivement ces problèmes.

		Durant ce stage, j'ai pu découvrir plusieurs aspects de l'analyse et la détection du cancer de la prostate grâce à ce sujet proposé en collaboration avec l'université de Tulane. J'ai aussi pu approfondir mes connaissances en reconstruction 3D, en manipulations d'espaces affines, ainsi qu'en interpolations en trois dimensions. En effet, ces thématiques étaient abordées dans le cursus de la spécialité IMAGINA, mais n'étaient pas autant détaillées. De plus, j'ai pu apprendre le fonctionnement de plusieurs techniques de visualisation en temps réel, et approfondir mes connaissances techniques sur le lancer de rayons, une technique utile en informatique graphique. Enfin, j'ai pris beaucoup de plaisir à aider Mme \textit{Faraj} à la soumission de son article scientifique à la convention IEEE VIS 2020, même si notre papier n'a pas été retenu.
	}

	\section{Travaux futurs}
	{
		Malgré les difficultés rencontrées engendrées par le changement d'organisation dû au COVID-19, nous avons réussi à créer une bonne base depuis lequel ce projet pourra avancer. Bien sûr, nous pouvons améliorer les performances du programme en parallélisant la génération de grilles, ou encore nous pouvons implémenter la méthode de génération de grilles hors-ligne comme discuté en section \ref{section:implementation}. Cela permettrait de s'affranchir des contraintes mémoire pour la génération de grille, et ainsi l'exécuter sur des grilles bien plus grandes.

		Pour ce qui est de la continuation directe du projet, nous pourrions implémenter une méthode de recalage d'images, afin de fusionner plusieurs prises de vues du même échantillon (voir figure \ref{img:multiple_image_captures}). Ou encore, nous pourrions continuer d'implémenter des méthodes de visualisation pour implémenter une visualisation multi-échelle, comme l'avait originellement prévu le stage. Nous pourrions également implémenter une méthode de déconvolution multi-vue, afin de l'adapter et de l'optimiser à notre besoin spécifique.
	}
	
	% etudes si deconv après reconstruction est judicieuse, étude si mélange de piles pour reconstruction bien réalisé (sert à refaire la vraie donnée) ...
	% etudes a long terme de l'analyse des glandes :
	%    - recalage
	%    - déconv multi-vue refaite ?
	%    - extraction de formes ?
	%    - classification ?
	%    - etude statistique de forme/taille/caractéristiques pour détection
}

% VIM modeline : do not touch !
% vim: set spell spelllang=fr :
