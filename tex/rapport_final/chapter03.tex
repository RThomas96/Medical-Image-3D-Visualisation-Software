% Contents of chapter 3
\chapter{Travail stage}\label{chapter:03:internshipwork}

\commentaire{
	\begin{enumerate}
		\item intro : chargements d'images, recherche de voisins et génération de grille
		\item travaux sur le chargement d'images\begin{enumerate}
			\item formats de sortie du microscope
			\item problème de tailles $\rightarrow$ downsampling
			\item recherche de librairies
			\item comparaison libtiff et tinytiff
			\item gestion de mémoire
			\item possibles travaux : buffer circulaire, chargement en multi-threading
		\end{enumerate}
		\item travaux sur la recherche de voisins\begin{enumerate}
			\item introduction au problème : à tulane ils le font pas
			\item rappel sur les paramètres du microscope
			\item présentation graphique du problème des voisins
			\item explication sur le besoin d'une méthode paramétrique (paramétrisable ?)
			\item explication de la méthode : espaces différents, relations mathématiques entre eux ... (cf le dessin de benjamin)
			\item implémentation de cette méthode
			\item présentation visuelle
			\item ouverture : on peut mieux avoir les voisins et donc interpoler comme il faut les voisins, donc grille de voxels !
		\end{enumerate}
		\item travaux sur la génération de grilles\begin{enumerate}
			\item recherche des voisins faite, mais toujours pas de notion de grille
			\item génération de grille : la facon simple
			\item génération de grille : prendre en compte les voisins, et interpoler pour avoir des résultats cohérents~\begin{itemize}
				\item l'interpolation nearest neighbor : tout bêta
				\item l'interpolation trilinéaire : bah c'est un peu mieux
				\item l'interpolation barycentrique : résiste au changement d'espace car espace défini par position des points du tétraèdre
			\end{itemize}
			\item le blending des deux stacks : mais pourquoi, et comment ?
			\item la sortie : à quoi elle sert ?
			\item implémentation (rapide, vu qu'on en parle extensivement avant)
		\end{enumerate}
	\end{enumerate}
}

% VIM modeline : do not touch !
% vim: set spell spelllang=fr :
