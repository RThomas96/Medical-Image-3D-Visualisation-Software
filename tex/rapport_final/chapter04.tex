% Chapitre 4 : travaux sur la grille de voxels de noura

\chapter{KidPocketFramework}% TODO : labeliser ce chapitre avec le nom complet

\commentaire{
	\begin{enumerate}
		\item Présentation de la méthode~\begin{itemize}
			\item Chargement de la texture
			\item Génération de maillage tétrahédrique à partir de maillage surfacique englobant
			\item (Rapide) Raymarching, Bresenham pour trouver voxel, et estimer normale et couleur
			\item Note : mentionner le fait que du coup, la complexité revient à la taille du framebuffer, non de la grille
			\item Mentionner le fait qu'on peut charger des grilles déformées afin de déformer les modèles en dehors de l'espace mémoire GPU
			\item Mentionner que ca va sortir à Vis 2020 si tout va bien
		\end{itemize}
		\item Présentation des travaux effectués~\begin{itemize}
			\item Mise à jour du code (dépoussiérage logiciel)
			\item Changement dans la gestion de la texture (vec4f $\rightarrow$ uchar + texture)
			\item Bug fixing pour grosses grilles
			\item Tests extensifs de la méthode pour publication scientifique~\begin{itemize}
				\item Tests de framerate
				\item Tests effectués sur grosses grilles
				\item Tests effectués sur petites grilles avec déformation
				\item Tests effectués à plusieurs résolutions
			\end{itemize}
		\end{itemize}
		\item Présentation des possibilités d'utilisation de la méthode dans notre cas~\begin{itemize}
			\item Une fois reconstruction effectuée, si taille grille générée < taille mémoire GPU, alors possiblité de voir interactivement le modèle
			\item Sinon, générer une grille plus petite et recommencer (pas besoin de super grande résolution pour voir les formes
				(Note : pas besoin de très grande résolution pour visualisation à l'oeil nu, on peut downsample)
		\end{itemize}
	\end{enumerate}
}
