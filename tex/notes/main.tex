\documentclass[a4paper]{article}

\usepackage{section}	% Sectionning the document
\usepackage{float}	% Floating boxes in document
\usepackage{graphicx}	% Images, and more
\usepackage{amsmath}	% Math symbols, commands, and environments

\title{Prostate3D : Notes, ramblings, and plans for the future}
\author{Thibault de Villèle}
\date{April 2nd, 2020 onwards}

\begin{document}
{
	\maketitle
	\clearpage

	\section{What should we do ?}
	{
	}
	\section{How should we do it ?}
	{
		%
	}
	\section{What's next ?}
	{
		%
	}
	\section{Raw notes, and useful snippets from papers}
	{
		\subsection{What we currently have}
		{
			We have SPIM imagery from Tulane in TIFF format saved a number of different ways :

			\begin{itemize} \label{list:different_image_types}
				\item Two stacks of 2000 pictures in 2048x2048 resolution \label{item:different_image_types:stacks}
				\item Two images : each with the entire image stack in one TIFF frame \label{item:different_image_types:single_image}
			\end{itemize}

			Each stack represents a full SPIM acquisition from one of the sensors of the microscope. The are angled 90 degrees relative to one another, and plus or minus 45 degrees relative to world coordinates.

			We should first of all, be able to determine a pixel's real coordinates in world-space, given an image, a stack, I and J positions (within the image) and voxel dimensions. Indeed, if we have a stack of images where the data is a sample being acquired using SPIM, then, given a stack $s$, an image count $d$ (for depth), a pixel position within that image $(i,j)$, we should be able to determine the 3D position $(x,y,z)$ of the point within the sample space.

			$$
			\text{If}~\left\{d, s, (i, j)\right\}~\text{then}~\exists~f(d,s,i,j)~\Rightarrow~(x,y,z)\label{math:existence_of_3D_coordinates}
			$$

			The only certainty we'll have is the angle between the two SPIM sensors : they will \textbf{always} be angled 90 degrees from one another, since it's a requirement of the method~\cite{cite_spim_explication_original}~.

			We can thus define some base assumptions :

			\begin{itemize} \label{list:reconstruction_assumptions}
				\item \label{item:reconstruction_assumptions:00} The world space coordinates are unimportant, we shall only focus on the sample space coordinates.
				\item \label{item:reconstruction_assumptions:01} The sample space is defined by the bounding box around the images, oriented the way they were taken in world space.
				\item \label{item:reconstruction_assumptions:02} The origin of the sample space is defined as the world space coordinate of the upper-left corner of the first image of the first stack.
				\item \label{item:reconstruction_assumptions:03} The stack shall be reconstructed as if the sample space origin is at the world origin, and shalld be aligned with world-space coordinates ($+Z$ in world space is the same as $+Z$ in sample space, same for other axes)
				\item \label{item:reconstruction_assumptions:04} The stack's Z coordinates will be the (up) vector.
				\item \label{item:reconstruction_assumptions:05} Each voxel's origin shall be placed at the world-space coordinate of the upper-leftmost pixel, at the image level, at the lowest Z-coordinate of the voxel.
				\item \label{item:reconstruction_assumptions:06} Each data within a voxel shall be assumed to be placed in the middle of the X-Y's voxel plane.
			\end{itemize}
		}
	}

	\bibliography{references}
	\bibliographystyle{apalike}
}
\end{document}
