\chapter[F.A.Q for users and developers]{How do I \ldots{} ? F.A.Q for users and developers}\label{text:04_faq}

\begin{comment}
	What will be included in this chapter :
		- User-facing stuff :
			- how to open/use the program, in general
				- use the help guide
				- say what can be loaded
				- describe interactions
				- describe viewing modes
				- describe output data (generation)
			- how to load data
				- what kind of filetypes are supported
				- what are some caveats/'gotchas' of the program
				- some of the options within the loading dialog
				- what the loading dialog should result in
			- how to generate grids
				- what the intent was to do something like this
				- how to use it
				- specify the process is blocking !!! no progressbar
				- what can be done with this
		- Dev-facing stuff :
				- How do I compile the program ?
					- The included third-party libraries
						- How to include them in the directory structure and in the compilation process
					- Compile the libs with the ./third_party/compile.{ps/sh} utilities
					- Use CMake
					- Learn about CMake flags !!!
				- How do I debug something ? -- Graphical
					- use renderdoc ! wonderful
				- How do I debug something ? -- Code
					- Learn a debugger (Qt's one is good)
					- Most code is documented, use this as a first step
					- Be warned, some things are multi-threaded so printf() might not always result in a useful thing.
				- How do I draw something ? [ NEEDED ??? ]
					- make sure context is bound
					- specify commands with an instance of QOpenGLFunctions (Scene, or within a drawable class)
					- some words about the composition of the widget, that all should be drawn and that Qt will do its stuff later
					- ping the reader back to the documentation of the pipeline
\end{comment}
