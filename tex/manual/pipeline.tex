\chapter{How the program works}

\begin{comment}
	What will be included in this section :
		- Very brief overview :
			- Open the program
			- Load a grid
			- Visualize it
			- Interact with it
			- Save it/close the program
		- Opening the program :
			- What gets created
			- What can you do right now ?
			- What does the code do in the background ?
		- Loading a grid :
			- What can you load ? (filetypes, types of data) -> for types of data, specify more info will be given in technical doc for the new grid api
			- what parameters can you specify ?
			- How to specify them ?
			- What does the code do in the background ?
		- Visualizing a grid :
			- What can you do ? (different viewing modes and how they work, quickly)
			- How do the things get drawn on screen ?
			- How do the different viewers get controlled ?
			- What does the code do in the background ?
		- Interacting with the grids on screen ?
			- What can you do ?
			- How does it work ?
			- What does the code do in the background ?
		- Generating/saving grids
			- What can you do with this ?
			- How does it work ?
			- What options/parameters can you set ?
			- What does the code do in the background ?
\end{comment}

\section{Overview of the system}
{
	% What is in this part :
	%
	%	- Very brief overview :
	%		- Open the program
	%		- Load a grid
	%		- Visualize it
	%		- Interact with it
	%		- Save it/close the program

	As explained beforehand, this program had a few very simple features, all in order to ease the visualization and management of huge datasets specifically to a doctor. You can load a big dataset, visualize it in real-time, and interact with it in a very simple fashion. Afterwards, you can save sub-parts of the grid in a new file.

	There are of course, some limitations to the kinds of images you can load in memory. Most notably, you can only load a few filetypes into the program. Those are \textsc{Tiff} files, \textsc{OME-Tiff} files, and \textsc{Dim/Ima} files.
}
