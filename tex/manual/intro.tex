\chapter{Introduction to the program}\label{text:01_intro}

\begin{comment}
	This section will include :
		- Why this program was created
		- What problems does it tackle / what are its features (very quickly)
		- How to read this manual ?
		- A few definitions
			- notably, the use of certain terms (dataset = 1/multiple 3D image(s))
\end{comment}

\section{What are the main goals of the program ?}\label{text:01_intro:01_goals}
{
	% Why was the program created, what does it do ?

	This program was created during my Master's internship to fit within the project "Visualization, management and processing of high-resolution medical images", done at the University of Montpellier. The work was continued during a small engineering position in 2020/2021. This is the result of those few months of work.

	Its original goal was to provide a simple way to generate high-resolution medical images issued from a \textit{di-SPIM} microscope. To this goal, it had to load a 3D image, downsampled enough to fit within the GPU's memory budget and deform it in order to visualize the resulting image interactively. It then had to allow the user to re-sample a part of the image, and save it to disk for later processing.

	While those tasks were implemented at the end of my internship, we later more features in the following engineering position. We added a real-time volumetric visualization method, allowing to see the sample in three-dimensions interactively. We also added the ability to load grids that would not fit within the memory budget, and downsample them on the fly at loading time in order to visualize bigger acquisitions. We also added the ability to generate the sub-parts of the image in an \guillemotleft~out-of-core~\guillemotright fashion, meaning it does not need to store neither the source image(s) or the resulting image(s) in memory, doing the processing directly from and to the disk.
}

\section{How to read this manual ?}\label{text:01_intro:02_howtoread}
{
	% How to read the manual : in any way you see fit.
	% Chapters have a main purpose, but can be read independently.
	% The code is _will_ change, and is 

	This manual is intended not only for users of the software, but also to provide the next developers of this software with a quick but precise overview of the processing pipeline, and of the underlying code.

	% TODO : add the references to the pipeline and code chapters here
	While chapters \ref{text:01_intro}\footnotemark and \ref{text:01_intro}\footnotemark will be more oriented towards the developers, the more curious users can peruse those parts of the manual in order to gain a better understanding of the inner workings of the code underneath.
	\footnotetext{TODO: change the references to the pipeline and code chapters of this manual.}

	A small note for the developers : this code was made heavily re-done between the internship and the engineering contract to simplify the development of new features, but some parts of it are still a work-in-progress (they will be designated as such in here). If you feel the need to refactor or change large parts of the codebase in order to either specialize or generalize a certain block of code, you're welcome to do it. However, I'd ask you update this documentation to mirror those changes.
}

\section{Glossary}\label{text:01_intro:03_definitions}
{
	For the sake of clarity, here are a few definitions of the words used within this manual, in no particular order :

	\begin{itemize}
		\item \textbf{dataset} : this can be used to design one or multiple images that make up an acquisition
	\end{itemize}
}

